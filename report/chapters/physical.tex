\section{Physical layer}
\label{physical}
This layer is all about physical properties of the hardware and the signal itself.\\
\todo{introduce, motivation, also explain setup, it's almost always with small LED unless otherwise stated}


%warmup
\subsection{Warmup time}

%These are stats for warmup only at dark
\begin{figure}[h]
\centering
\includegraphics[height=140px]{img/warmup1}
\includegraphics[height=140px]{img/warmdown1}
\caption{LED warmup and cooldown times.}
\label{fig:warmup}
\end{figure}

\todo{stats for warmup}
\todo{table with times to warmup from x to y, eg 0-0, from 0 to 10, 0-20 ,0-30, ..., 10-0 10-10; then discuss meaningful milestones }
The speed of light transmission depends in primis on the speed at which the light itself can be turned on and off.
Figure \ref{fig:warmup} shows the warmup times for the low power LED over multiple instances, meaning the time that it takes for turning the light completely on from completely off, and vice versa.
These measurements are bound to the reception rate of the system, which will be discussed in section \ref{recrates}.
Table \ref{tab:warmup} shows the times for the LED to switch between specific brightness levels.
Each row represents the time to reach the level on each column, for example the first row represents the time to reach any brightness level starting from 0\% brightness.
The table works both ways, meaning it shows the time for the warmup as well as the time for the cool down of the LED.
The last row shows how long it takes to reach any level from a completely ON state, meaning 100\% of brightness.

\begin{table}[hbt]
\centering
  \begin{tabular}{ l | c c c c c c c c c c c}
    & 0\% & 10\% & 20\% & 30\% & 40\% & 50\% & 60\% & 70\% & 80\% & 90\% & 100\% \\
    \hline
0\% & - & 0.3 & 0.66 & 1.08 & 1.52 & 2.12 & 2.82 & 3.78 & 5.38 & 10.4 & 50.66 \\
10\% & 55.88 & - & 0.36 & 0.78 & 1.22 & 1.82 & 2.52 & 3.48 & 5.08 & 10.1 & 50.36 \\
20\% & 61.86 & 5.98 & - & 0.42 & 0.86 & 1.46 & 2.16 & 3.12 & 4.72 & 9.74 & 50.0 \\
30\% & 63.76 & 7.88 & 1.9 & - & 0.44 & 1.04 & 1.74 & 2.7 & 4.3 & 9.32 & 49.58 \\
40\% & 64.9 & 9.02 & 3.04 & 1.14 & - & 0.6 & 1.3 & 2.26 & 3.86 & 8.88 & 49.14 \\
50\% & 65.76 & 9.88 & 3.9 & 2.0 & 0.86 & - & 0.7 & 1.66 & 3.26 & 8.28 & 48.54 \\
60\% & 66.4 & 10.52 & 4.54 & 2.64 & 1.5 & 0.64 & - & 0.96 & 2.56 & 7.58 & 47.84 \\
70\% & 66.98 & 11.1 & 5.12 & 3.22 & 2.08 & 1.22 & 0.58 & - & 1.6 & 6.62 & 46.88 \\
80\% & 67.48 & 11.6 & 5.62 & 3.72 & 2.58 & 1.72 & 1.08 & 0.5 & - & 5.02 & 45.28 \\
90\% & 67.92 & 12.04 & 6.06 & 4.16 & 3.02 & 2.16 & 1.52 & 0.94 & 0.44 & - & 40.26 \\
100\% & 68.28 & 12.4 & 6.42 & 4.52 & 3.38 & 2.52 & 1.88 & 1.3 & 0.8 & 0.36 & - \\
  \end{tabular}
  \caption{Warmup times of the LED, for specific levels of brightness.}
  \label{tab:warmup}
\end{table}

\todo{stats for big bulb?}

%distance
\subsection{Distance}
\todo{stats for different distances}
test for different distances, talk about brightness-distance (dark, no interference)

%angle
\subsection{Angle}
\todo{stats for different angles}
test the light with different angles to the sensor (dark)

%power ac vs dc
\subsection{Power source}
\todo{talk about AC vs DC, discussion about power vs brightness vs distance vs speed}
\begin{figure}[h]
\centering
\includegraphics[height=140px]{img/warmup2}
\includegraphics[height=140px]{img/warmdown2}
\caption{AC bulb warmup time, and phases of alternating current.}
\label{fig:warmupAC}
\end{figure}
\todo{remove outlier in the picture of ac?}
