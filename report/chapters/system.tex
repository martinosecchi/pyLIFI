
[In the next chapters I describe the overall characteristics and limitations that a VLC system should present according to my findings, identify the variables involved and the set of parameters necessary for it to work\\
This chapter should provide a useful starting point for whoever plans to implement one in practice]\\
What does one need to make his own VLC system?\\
%Structure of a VLC system: logical layer, control layer, physical layer. Analysis and characteristics of the three.


%logical abstraction
Every visible light communication system shares the same underlying structure.
Communication needs to be established between two ends, one end that acts as a transmitter and the other that acts as a receiver.
There are at least three levels of abstraction needed to transform a digital information into a signal that can be sent and received, and these are: 
\begin{itemize}
\item logical layer
\item control layer
\item physical layer
\end{itemize}
Each of these levels present different characteristics that combined form the overall performance of the system.\\

%physical
In VLC, the transmitter end of the communication operates on a light emitter to vary a property of the light depending on the modulation scheme used (see \ref{modulschemes}), like brightness or colour.
The receiver on the other end needs to be able to detect such variations in a measurable manner.
This level can be described as the \textbf{physical layer} of the communication.
At this level, the system's performance can be influenced by physical properties of the hardware, like maximum brightness of the light emitter, speed in switching ON/OFF state and warmup time, but also by other factors, like distance between the two ends, the medium of transmission, the angle of incidence of the light between transmitter and receiver and so on.\\

%control
This variations need to be controlled and organised to carry specific information.
A \textbf{control layer} is necessary in order to make the link between logical information and variation of physical property.
This layer is of critical importance for the performance of any system, since it influences the frequency of the physical variations which results in the rates for transmission and reception.
Factors that are to be considered in this layer are directly linked to the hardware components, namely any characteristic that influences the overall speed of transmission/reception, like clock rates of the micro controllers, speed of the transistors used and so on.\\

%logical
The outer most level of abstraction is the \textbf{logical layer}, where information is handled and manipulated at a software level.
For a transmitter, this layer produces the instructions to pass on to the control layer in order to generate a signal given a specific information. 
This process can be seen as the logical encoding of the information, ready to be transferred and become physical encoding.
At the opposite end of the transmission, the logical layer of a receiver is given data about physical variations measured by the sensor(s) used in the system, and needs to reconstruct and interpret such data back to the original information.
Contrary to the previous layers, the performance of the logical layer doesn't rely much on the hardware and the physical characteristics of the components, but rather on the software techniques and algorithms that implement it.
A well structured logical layer can even add more reliability to an otherwise uncertain medium of communication, as will be seen in later chapters.

\todo{do a graph of logical control and physical.}

%prototype architecture
\subsection{Experimental setup}

In order to verify feasibility, investigate characteristics and test performance of general VLC systems, a prototype system has been developed.
The system is composed of two main modules: a transmitter module, and a receiver.
The transmitter module uses On Off Keying with Manchester Coding (see \ref{modulschemes})  to convey signals with light.
In order to allow testing, this module takes some arbitrary input from a user, encapsulates the information and encodes it to produce variations in light intensity, through the control of a light emitter.
The receiver side measures the light variations through the use of a photoresistor, or light sensor, reconstructing and interpreting the sensor data back into the original messages. 
Each module includes different components, listed in section \ref{components}.
Fig. \ref{fig:sys over} shows an overview of the prototype architecture.

\begin{figure}
\centering
\includegraphics[height=200px]{img/sysover}
\caption{Prototype system.}
\label{fig:sys over}
\end{figure}

\todo{in the experimental setup, it's considered almost exclusively low power DC LED}

%transmitter
\subsubsection{Transmitter}
Transmission starts from a terminal, where a client can input messages as strings. These are then encapsulated into Protocol Data Units (PDUs) and sent to a micro controller through a serial connection.
In this case, the micro controller that was used for transmission is an Arduino board.
The board encodes the received bytes into Manchester Code, and then controls the light signal by switching on and off the emitter accordingly.
A light emitting diode is the furthest end of the transmitter module.
For this part, multiple setups have been tried.
The fastest emitter that has been tested is single low power LED connected to the board and powered directly by it, which allows very fast switching.
A second setup has also been tested with a commercial LED bulb powered by an external power source (mains electricity), and controlled by the board through a switcher. 

%receiver
\subsubsection{Receiver}
%Sending light signals is not too complicated, what really is the core of the project is to read such signals and interpret the sensor data correctly.
The signal is received by a photodiode read as analog input by a micro controller.
On the board, values are software-filtered to reduce noise, and sent to the receiving terminal with a timestamp.
The board and the terminal are connected through a serial connection.
Some controllers could read directly analog input and perform the remaining processes to interpret the signal, with enough computational power.\\ 
For this application, the final terminal is a Raspberry Pi, since it has a good power/size ratio for the designed purpose and allows a simple prototyping process. The micro controller in between the sensor and the terminal is necessary to read and forward analog data, which is not possible directly from the Raspberry Pi.\\
On the final terminal, the variations of sensory data are interpreted as sequences of digital 0s and 1s, finally decapsulated and decoded back into a message.

%components
\subsubsection{List of components}
\label{components}
Each module of the prototype system is composed of different components, listed in the following.\\
Transmitter:
\begin{enumerate}
\item personal computer as main terminal for user input and information processing
\item Genuino Uno micro controller board, based on ATmega328P \cite{genuinouno}
\item Light emitting diodes:
\begin{enumerate}
\item low power LED, 5V 
\item commercial LED bulb, 230 V AC, 3 W, 240 lumen
\end{enumerate}
\item Solid State Relay for Arduino, 5V activation, 240V load
\end{enumerate}
Receiver:
\begin{enumerate}
\item Photoconductive Cell VT900
\item Genuino Yun micro controller board, based on ATmega32u4 \cite{arduinoyun}
\item Terminals:
\begin{enumerate}
\item personal computer for test of communication
\item Raspberry Pi 3 Model B to be applied on a mobile robot
\end{enumerate}
\end{enumerate}

\subsubsection{Wiring}
\todo{add wiring? diagram for small setup and most importantly setup with ssr and big bulb}

