[This is some introduction.
Here I introduce what visible light is, and the vision of how it could be used in general and in the specific scenarios for robot communication, environment to robot and robot to robot and why it would make sense to suggest it.]

Visible Light Communication (VLC) is a type of data communication achieved with the transmission of signals in the spectrum of visible light.
%The focus of this project is to implement a VLC based system to be applied in the context of mobile robotics, in which robotic agents are able to receive and react to such signals.\\
%After the design and implementation of the prototype of a VLC communication system, main focus of the project, such prototype will be applied in a scenario where communication will assist mobile robots in performing simple tasks and navigation through the environment.\\
%situated
This form of communication, compared to more classic ones, could potentially be beneficial in the context of mobile robotics among others, for the reason that light is a form of \textbf{situated} communication, where the message is not separated from the physical environment in which it has meaning.\\
This allows to transmit a message that includes additional information embedded in the medium of transmission rather than its content.
Particularly relevant for mobile robotics is the information relative to localisation and direction, that can just be seen as the source of the light instead of it being encoded in the message.
In this way, the content of the message and the properties of the transmission combined together produce the information to be received. \\
%applications
Such a technology could be used in it simplest form in the context of mobile robotics to provide an aid for environment navigation, or a form of remote control for agents' movement or for assistance in the execution of specific tasks. 
These are a few examples of mono directional, environment-to-robot communication systems.
The technology could also be used in a slightly more complex scenario of robot-to-robot 1- or even 2-way communication between multiple agents to enable cooperation.
%design
The design of a prototype system will help analyse the properties of a generic VLC system, its performances and results in the given applicative scenario.
The design process will be focused on simplicity and inexpensiveness where possible, by using mostly off-the-shelf and easily obtainable components.\\
...\\
%lifi and optical
In recent years optical communication has seen a great spread mostly thanks to optical fibre.
As opposed to optical fibre communication, VLC is meant to be propagated wirelessly over an open space.
By modulating a light source, like could be a common LED bulb, through a controller it is possible to encode messages from digital form into light signals.
The simplest example of this encoding is a binary representation of data through light, where the presence of light represents a binary 1 and its absence a 0.
This form of communication is a simplified variation of the technology known as Li-Fi, a term that stands for Light-Fidelity.

\todo{much more introduction}
\todo{mention cheap and simple components, getting it to work is simple, getting it to work fast and far is very hard/ expensive}
\todo{mention fiber vs open space?}